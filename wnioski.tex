Celem pracy było zaprojektowanie społecznościowego systemu, dla którego ideą było udostępnianie lokalizacji osób w promieniu 10 km od miejsca aktualnego przebywania. 
Stworzona została aplikacja internetowa, która z połączeniu z aplikacją na platformę \emph{Android} tworzy jeden projekt. W trakcie realizacji skupiono się przede wszystkim na części \emph{webowej} i stronie serwera, co zajęło najwięcej czasu. Aplikacja \emph{Android} w zgodzie z wymaganiami, spełnia swoją podstawową funkcję, czyli przesyła położenie zalogowanego na serwerze użytkownika. Przesłane dane mogą być udostępniane innym użytkownikom serwisu w zgodzie z zaznaczoną prywatnością.

W trakcie ustalania środowiska projektu zdecydowano, że system aplikacji społecznościowej będzie stworzony w języku \emph{Python} z zastosowaniem \emph{frameworka Django}.
Jest to stosunkowo nowe narzędzie na rynku darmowych rozwiązań \emph{Open Source}, jednak bardzo rozbudowane i wydajne w użyciu, o czym można było się przekonać podczas pisania projektu.  Motywem przewodnim \emph{Django} jest realizacja zasady \emph{DRY} (\emph{Don’t repeat yourself}), czyli nie powtarzaj się. Zasada została wykorzystana tam, gdzie tylko było to możliwe, również w trakcie pisania aplikacji na system \emph{Android}. Programowanie zgodnie z \emph{DRY} wymusiło pisanie takich funkcji i metod, które można wykorzystać ponownie w innym miejscu.

Implementacja projektu przebiegała bez większych problemów. Zastosowany wzorzec architektoniczny \emph{MVC}, w wersji jaką preferują twórcy frameworka \emph{Django}, pozwolił na organizację kodu w czytelny sposób. Wystarczyło raz zaprogramować modele aplikacji, by później zająć się wyłącznie projektowaniem funkcjonalności. W trakcie implementacji aplikacji na platformę \emph{Android} wykorzystywano potencjał i szerokie wsparcie dla języka \emph{Java} co również umożliwiło sprawne stworzenie aplikacji.

W początkowej fazie projektu aplikacji internetowej wykorzystano technikę \emph{test-driven development}. Napisane testy sprawdziły podstawową funkcjonalność. Później stworzone testy wykorzystano ponownie w trakcie opisywania innych przypadków testowych dla napisanych już funkcjonalności. \emph{Framework Django} został wyposażony we własne narzędzie testowania aplikacji bazujące na standardowych rozwiązaniach języka \emph{Python}, co znacznie uprościło sam proces testowania i pisania testów.
Po stronie projektowania aplikacji \emph{Android} wykorzystano środowisko \emph{JUnit} wchodzące w skład edytora \emph{Eclipse}, co pozwoliło na równoległe testowanie i pisanie funkcjonalności.

Aplikację internetową wdrożono na serwerze \emph{Apache}, a jako interfejs bramy sieciowej dla języka \emph{Python} wybrano \emph{WSGI} ze względu na wystarczające wsparcie w tym projekcie dla aplikacji napisanych w \emph{Django}. W przypadku instalacji aplikacji wybór wykorzystanego \emph{frameworka} również okazał się poprawny. Przed założeniem tabel w bazie, wystarczyło tylko uzupełnić plik konfiguracyjny aplikacji o dane używane do połączenia z bazą. Po uruchomieniu skryptu, który zakłada potrzebne tabele, aplikacja jest gotowa do pracy.

Program na system operacyjny \emph{Android} wystarczyło udostępnić w \emph{Internecie} i wygenerować dla odnośnika kod \emph{QR}, który znakomicie spełnia się w tej roli. Instalacja programu ograniczyła się tylko do zeskanowania kodu, pobrania i instalacji.

W trakcie realizacji zadania pomocna okazała się społeczność \emph{Open Source}. Wykorzystana technologia \emph{Python} posiada szeroką społeczność, zawsze chętną do pomocy w razie problemów z językiem. Podobnie z \emph{frameworkiem} \emph{Django}, który jest bardzo dobrze udokumentowany na stronie projektu, co również miało wpływ na czas realizacji zadania. Aplikacja na platformę \emph{Android} została zaprojekowana języku \emph{Java}. Jest to język prosty i również bardzo dobrze udokumentowany. Ponadto \emph{Google} stworzył własną stronę dla projektu \emph{Android}, z pełnym opisem klas i metod. Wykorzystano również inne projekty na otwartej licencji, które posłużyły jako wzór dla zrealizowanego zadania.