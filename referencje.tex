%referencje
%\bibliographystyle{abbrv}

\renewcommand*{\refname}{} % This will define heading of bibliography to be empty, so you can...
\section*{Referencje}\label{sec:referencje}     % ...place a normal section heading before the bibliography entries.
\addcontentsline{toc}{section}{Referencje}
\bibliographystyle{ieeetr}
\begin{thebibliography}{99}
\bibitem{iphone}
	\emph{iPhone} - Smartfon opracowany i wyprodukowany przez firmę \emph{Apple Inc.}. 
	\url{www.iphone.com}
\bibitem{apple}
	Apple Inc. - Amerykańska korporacja zajmująca się projektowaniem i produkcją elektroniki użytkowej, 
	oprogramowania i komputerów osobistych. Mieści się w \emph{Cupertino} w Kaliforni (USA). 
	\url{www.apple.com}
\bibitem{google}
	\emph{Google Inc.} - Amerykańskie przedsiębiorstwo z branży internetowej. Producent systemu operacyjnego \emph{Android}.
	\url{www.google.com}
\bibitem{android}
	\emph{Android} - System operacyjny dla urządzeń mobilnych.	
	\url{www.android.com}
\bibitem{python}
	\emph{Python} - Interpretatywny, obiektowy język programowania.
	\url{www.python.org}
\bibitem{java}
	\emph{Java} - obiektowy język programowania wysokiego poziomu.
	\url{http://www.java.com/en/java_in_action/}	
\bibitem{apache}
	\emph{Apache2} - Najpopularniejszy, otwarty serwer \emph{HTTP}.
	\url{http://httpd.apache.org/}
\bibitem{opensource}
	\emph{Open Source} - Ruch promujący otwarte oprogramowanie.
	\url{http://www.opensource.org/}
\bibitem{postgresql}
	\emph{PostgreSQL} - System zarządzania bazami danych \emph{PostgreSQL}.
	\url{www.postgresql.org}
\bibitem{modwsgi}
	\emph{MOD WSGI} - Adapter języka \emph{Python} dla serwera \emph{Apache}.
	\url{http://code.google.com/p/modwsgi/}
\bibitem{sublime_text}
        \emph{Sublime Text 2} - Edytor tekstu i kodu aplikacji
	\url{http://www.sublimetext.com/}
\bibitem{vim}
	\emph{VI Improved} - Wieloplatformowy edytor tekstowy należący do grupy wolnego oprogramowania o otwartym kodzie źródłowym.
	\url{http://www.vim.org/}
\bibitem{eclipse}
	\emph{Eclipse} - Środowisko programistyczne.
	\url{www.eclipse.org/}
\bibitem{android-sdk}
	\emph{Android SDK} - Narzędzia programistyczne, biblioteki, \emph{debugger} kodu, emulator i dokumentacja.
	\url{http://developer.android.com/sdk/index.html}
\bibitem{adb}
	\emph{Android Debug Bridge (ADB)} - Narzędzie wchodzące w skład \emph{Android SDK}, narzędzie lini poleceń typu klient-serwer pozwalające na komunikację z emulatorem urządzenia lub z urządzeniem z systemem operacyjnym \emph{Android}.
\bibitem{ubuntu1110}
	\emph{Linux Ubuntu 11.10} - System operacyjny typu \emph{Open Source}.
	 \url{www.ubuntu.com}
\bibitem{androidmarket}
	\emph{Android Market} - Internetowy sklep \emph{Google} z aplikacjami na urządzenia działające pod kontrolą systemu operacyjnego \emph{Android}.
	\url{http://market.android.com}	
\bibitem{wersje_android}
	\emph{Wersje platformy Android}\\
	\url{http://developer.android.com/resources/dashboard/platform-versions.html}	
\bibitem{google_maps}
	\emph{Google Maps} - Serwis internetowy umożliwiający wyszukiwanie obiektów, oglądanie map i zdjęć lotniczych z powierzchni Ziemi.
	\url{www.maps.google.pl}

\end{thebibliography}



