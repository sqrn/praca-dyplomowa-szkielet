%implementacja-logirej
Implementacja logowania i rejestracji użytkowników to pierwsze zadanie zrealizowane podczas pisania projektu. W trakcie poszerzania możliwości serwisu, prezentowany kod często ulegał refaktoryzacji. Poniższy rozdział prezentuje ostateczną wersję kodu modułu rejestracji i logowania.

Pierwsza prezentowana funkcjonalność została napisana w języku \emph{Python} z zastosowaniem \emph{frameworka Django}, czyli tak jak większość opisywanej aplikacji. Mając na uwadzę stosowany wzorzec architektoniczny \emph{MVC}, w pierwszej kolejności zaprezentowany będzie model, a po nim logika biznesowa.

Najważniejszy model aplikacji, a zarazem profil użytkownika to klasa \texttt{UserProfile} (listing \ref{kod:UserProfile}).\\

\lstinputlisting[
  label=kod:UserProfile, 
  caption=Klasa modelu UserProfile]
  {src/UserProfile.py}

\vspace{1em}
Zaprezentowany kod ma swoje przełożenie na budowę tabel w bazie danych. \emph{Django} wykorzystuje modele po ich napisaniu do stworzenia tabel o takiej samej nazwie jak nazwa klasy modelu poprzedzona nazwą aplikacji. Parametry klasy \texttt{UserProfile} to kolumny tabeli.

Komentarza wymaga parametr \texttt{user}, \texttt{friends}
oraz 
\texttt{privacy}. 
Są to odwołania do kluczy obcych innych tabel utworzonych przez inne modele \emph{Django}. Parametr \texttt{user} to obiekt klasy wbudowanej \texttt{User}. Jest używany wszędzie tam gdzie wymagany jest obiekt użytkownika lub \emph{ID} encji w bazie danych. Parametr \texttt{friends} jest referencją do innej tabeli, utworzonej podczas tworzenia tabeli dla modelu \texttt{UserProfile}. Przechowuje ona relacje zawartej znajomości między użytkownikami serwisu. I ostatni parametr - \texttt{privacy} to referencja do klasy \texttt{UserPrivacy}, która odpowiada za prywatność użytkowników (listing \ref{kod:UserPrivacy}).\\

\lstinputlisting[
  label=kod:UserPrivacy, 
  caption=Klasa modelu UserPrivacy]{src/UserPrivacy.py}

\vspace{1em}
Za logikę aplikacji odpowiadają widoki. Poniżej prezentacja kodu aplikacji \texttt{accounts}, który realizuje rejestrowanie użytkowników (listing \ref{kod:register}). \\

\vspace{1em}
Metoda \texttt{register} to zwykła funkcja napisana w języku \emph{Python}. Przyjmuje ona tylko jeden parametr, którym jest żądanie (\texttt{request}) przesłane przez użytkownika. Jest to obiekt klasy \texttt{HttpRequest}, który przechowuje dość sporo informacji. Najważniejsza dla funkcji jest tablica \texttt{POST}, która przechowue dane wpisane do formularza. 
Za formularz rejestracji odpowiada klasa \texttt{RegisterForm} (listing \ref{kod:RegisterForm}).\\

\lstinputlisting[label=kod:RegisterForm, caption=Formularz rejestracji nowego użytkownika]{src/RegisterForm.py}

\vspace{1em}
Formularz \texttt{RegisterForm} posiada cztery parametry, które są obiektami klas wbudowanych w \emph{Django}. Sam formularz to również klasa i dziedziczy po głównej klasie \texttt{Form} z pakietu \texttt{django.forms}. Dzięki generalizacji, klasa formularza ma dostęp do trzech niezbędnych metod - \texttt{is\_valid}, \texttt{clean} oraz \texttt{save}. Pierwsza jest używana w funkcji \texttt{register} z poprzedniego kodu. Wywołuje ona metodę \texttt{clean} w klasie \texttt{Form}, która weryfikuje poprawność wypełnionego formularza i zwraca błąd, gdy formularz jest źle wypełniony. Metodę tą można nadpisać i dostosować do własnych wymagań, jednak w tej klasie wykorzystano tylko \texttt{is\_valid}, która zwraca \texttt{True} lub \texttt{False}.

Kolejna metoda - \texttt{save} - dodatkowo waliduje poprawność wprowadzonych danych, w sposób wymagany do poprawnego działania serwisu. Jeżeli weryfikacja napotka błąd, do widoku zostanie zwrócony \emph{False}, a na stronie zostanie wyświetlony odpowiedni komunikat. Jeżeli weryfikacja nie napotka żadnych problemów, zostaje utworzone nowe konto użytkownika, nowy profil oraz prywatny katalog na dysku serwera.

Pola formularza będą wyświetlone na stronie, bez znaczenia na wynik walidacji danych. Funkcja widoku zawsze tworzy instancję formularza, która jest zwracana jako parametr do szablonu \emph{template} \texttt{register.html}. Tam traktowany jest jak obiekt w specjalnym języku szablonów \emph{Django}. Na przykład wywołanie \texttt{\{\{form.email\}\}} zwróci na stronie formularz \emph{input} dla pola \emph{e\-mail}.

W kolejnej części znajduje się opis kodu odpowiedzialnego za logowanie użytkownika. W tym przypadku zaprezentowany zostanie wyłącznie kod logiki, ponieważ część modelu jest taka sama jak w przypadku rejestracji. Poniżej znajduje się kod widoku - funkcja \texttt{log\_in} z widoku aplikacji \texttt{accounts} (listing \ref{kod:log_in}).\\

\lstinputlisting[label=kod:log_in, caption=Funkcja logowania]{src/log_in.py}

\vspace{1em}
W tym przypadku, weryfikacja danych wymaganych do poprawnego działania aplikacji realizowana jest bezpośrednio po stronie widoku. Funkcja wywołuje formularz \texttt{LoginForm} (listing \ref{kod:LoginForm}).\\

\lstinputlisting[label=kod:LoginForm, caption=Formularz logowania]{src/LoginForm.py}

\vspace{1em}
Rolę jaką pełni funkcja widoku to przede wszystkim dopisywanie do obiektu \texttt{messages} wiadomości, które wyświetlone zostaną na stronie osoby wywołującej stronę logowania. Obiekt \texttt{messages} to element \emph{frameworka} o tej samej nazwie, który jest częścią \emph{Django}. Jego działanie jest proste i w zaprojektowanej pracy, realizuje przesyłanie wiadomości do użytkownika, które wyświetlone będą na stronie.

Po przeprowadzonej weryfikacji, w odpowiedzi zostanie zwrócony obiekt formularza do parametru \texttt{form}, uzupełniony o wiadomość w obiekcie \texttt{messages} - obiekt ten zwrócony jest do żadania w lini 33. 

Implementacja modułu logowania i rejestracji może z początku wyglądać dziwnie, mając w pamięci standardowe podejście do tworzenia i wyświetlania formularzy na stronie internetowej, na przykład w języku \emph{PHP}. Formularze \emph{Django} mają być przede wszystkim tak napisane, aby ich kod nie był powtarzany (zgodnie z zasadą \emph{DRY}) i po krótkiej analizie kodu, można dojść do wniosku, że framework robi to w sposób czytelniejszy. Dodatkowo \emph{Django} wyposarza programistę w dodatkowe narzędzia wspomagające walidację i prezentację danych na stronie.

Możliwość logowania na serwerze została zaimplementowana także w aplikacji na system \emph{Android}. Poniżej znajduje się kod klasy \texttt{Main}, gdzie realizowana jest podstawowa logika programu na platformie \emph{Android}.
Klasa \texttt{Main} posiada klasy \texttt{User} oraz \texttt{Geolocation} (listing \ref{kod:android-main}). Dziedziczy po głównej klasie \emph{Android} - \texttt{Activity} i implementuje interfejs \texttt{OnClickListener} udostępniający metodę \texttt{onClick} wywoływaną gdy dotknięty zostanie przycisk \texttt{loginButton}. Logika tej klasy realizowana jest w metodzie \texttt{onClick}, gdzie ładowane są ustawienia oraz uruchamiane jest logowanie do serwera.\\

\lstinputlisting[label=kod:android-main, caption=Klasa Main aplikacji \emph{Android}]{src/android-main.java}

\vspace{1em}
Metoda \texttt{onClick} tworzy obiekt klasy \texttt{User} o nazwie \texttt{user}, której konstruktor przyjmuje dwa parametry - \emph{e\-mail} oraz hasło pobrane z konfiguracji \texttt{preferences}. Następnie dla obiektu \texttt{user} wywołana zostaje metoda \texttt{tryLogin}, która loguje użytkownika na serwerze. W odpowiedzi aplikacja otrzyma kod \emph{200} albo \emph{401}.

Logowanie po stronie serwera realizuje funkcja \texttt{mlogin} z widoku aplikacji \texttt{mobile}. Jest ona podobna do funkcji \texttt{log\_in}, z tą różnicą, że \texttt{mlogin} przyjmuje dane w formacie \emph{JSON}. Po odczytaniu ich, następuje logowanie na serwerze lub zostanie zwrócony bład w razie niepowodzenia. Poniżej znajduje się kod funkcji \texttt{mlogin}.\\

\vspace{1em}
W przypadku \texttt{mlogin}, funkcja wykorzystuje formularz \texttt{LoginForm} z aplikacji \texttt{accounts}. Dane \emph{JSON} odczytane z przesłanego żądania \emph{HTTP} są przekazane do konstruktora klasy \texttt{LoginForm}. Jeżeli przesłane są niepoprawne, zwrócony zostanie status \emph{401}. Zaletą wykorzystania wcześniej napisanego formularza, jest wykorzystanie metody \texttt{clean}, która waliduje przesłane dane i zabezpiecza serwer przed niebezpiecznymi atakami.

Logowanie w serwisie ze strony aplikacji na platformę \emph{Android} i wewnątrz serwisu zostało zaprezentowane. Prezentowany kod realizuje swoje zadanie tak jak projektowano. Walidacja danych jest wykonywana poprawnie, a szczegóły walidacji oraz przyjęte przypadki testowe będą prezentowane w dziale \ref{sec:testy}.