%architektura-postgresql
To relacyjna, wydajna baza danych, początkowo rozwijana na Uniwersytecie Kalifornijskim w \emph{Berkeley} pod nazwą \emph{Ingres}. Później, aby uzgodnić zgodność ze standardem \emph{SQL}, zmieniono nazwę na \emph{PostgreSQL} (czasami nazywana również \emph{Postgres}). Wspomniany system baz danych obok innych darmowych rozwiązań takich jak \emph{MySQL} czy \emph{Firebird} wyróżnia się wysoką wydajnością oraz możliwościami jakie oferuje \cite[s.\pageref{sec:referencje}]{mysql,firebird}. Jednym z atutów \emph{Postres} jest możliwość pisania własnych poleceń składowanych w różnych językach programowania. Podobne rozwiązanie stosuje komercyjny \emph{Oracle}, gdzie można projektować funkcje przy pomocy \emph{PL/SQL}, a od wersji 8i również w języku \emph{Java} \cite[s.\pageref{sec:referencje}]{oracle}.